\section{An\'alisis de requerimientos}
Los requisitos planteados para la implementaci\'on del sistema son los siguientes.

\subsection{Requerimientos Funcionales}

\begin{itemize}
\item{RF01:} Permitir el ingreso de usuarios al sistema por medio de una verificaci\'on. 
\item{RF02:} Clasificar la conversaci\'on a partir de diversos patrones presentados.
\item{RF03:} Las conversaciones se guardan en archivos xml.
\item{RF04:} El sistema realiza el pre-procesamiento mediante el stemming, la lematisaci\'n y eliminar  stop words para crear una lista de palabras junto con su frecuencia. 
\item{RF05:} El sistema realizara la clasificaci\'on de la conversaci\'on mediante un metodo de toma de decisiones.
\end{itemize}

\subsection{Requerimientos No Funcionales}

\begin{itemize}
\item{RNF01:} El modulo del chat debe accederse a trav\'es de un ambiente Web. 
\item{RNF02:} El sistema debe estar separado en modulos independientes.
\item{RNF03:} El sistema debe ser altamente escalable, es decir debe de poderse agregar nueva funcionalidad sin perder la calidad y el funcionamiento que ya se ha alcanzado.
\end{itemize}

\subsection{Especificaciones Suplementarias (No Funcionales)}
 
Confiabilidad
 
El sistema debe asegurar al usuario un funcionamiento adecuado, ya que no puede permitirse ning\'un tipo de error cuando se clasifica la conversaci\'on. 

Adaptabilidad 

El sistema que se desea implementar debe ser lo suficientemente adaptable a cualquier navegador Web y sistema operativo sobre el que se corra la aplicaci\'on. 

Escalabilidad 

Debe de aceptar cualquier tipo de extensi\'on de la funcionalidad. Adem\'as de aceptar en el futuro integraci\'on con sistemas ya existentes.