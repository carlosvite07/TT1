%---------------------------------------------------------
\section{Introducci\'on}

Con la llegada de las computadoras personales en el a\~no 1970, surgi\'o la demanda de aplicaciones pr\'acticas que utilizaran reconocimiento de patrones.

El reconocimiento de patrones es una disciplina cient\'ifica cuyo objetivo es la clasificaci\'on de objetos dentro de categor\'ias o clases. Dependiendo de la aplicaci\'on, estos objetos pueden ser im\'agenes, se\~ nales o cualquier tipo de entidades medibles que necesiten ser clasificadas.\cite{reconoci} El reconocimiento de patrones es una parte integral de la mayor\'ia de los sistemas de inteligencia artificial construidos para la toma de decisiones.
        
Una subdisciplina de  la inteligencia artificial es el procesamiento de im\'agenes  cuyo t\'ermino es utilizado para describir a la funci\'on implementada por software o hardware que permite analizar y clasificar im\'agenes la cual emplea teor\'ia del reconocimiento de patrones. \cite{intelig} \\

El presente documento contiene el an\'alisis y la documentaci\'on del Trabajo Terminal 2015-B032 "Sistema de identificaci\'on de art\'iculos utilizando reconocimiento de patrones.", el cual pretende implementar t\'ecnicas de las dos disciplinas descritas anteriormente.
\pagebreak