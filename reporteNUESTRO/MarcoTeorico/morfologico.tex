\section{An\'alisis morfol\'ogico}

En el espa\~nol existe una enorme cantidad de palabras que pueden desempe\~nar diferentes funciones gramaticales. El an\'alisis de un texto producir\'a una gran multiplicidad de combinaciones posibles.

El an\'alisis morfol\'ogico consiste en la detecci\'on de la relaci\'on que se establece entre las unidades m\'inimas que forman una palabra: reconocimiento de sufijos o prefijos. Este nivel de an\'alisis  mantiene una estrecha relaci\'on con el l\'exico. \cite{elprofesionaldelainformacion}

El an\'alisis morfol\'ogico consiste en determinar la forma, clase o categor\'ia gramatical de cada palabra de una oraci\'on. \cite{morfo}

\begin{center}

\begin{table}[h]

\begin{tabular}{|l|p{75mm}|l|l|p{50mm}|}

\hline

Sustantivo &  Masculino o Femenino, Singular o Plural, Com\'un o Propio, Concreto o Abstracto, Individual o Colectivo. Si lleva Sufijo, puede dar idea de aumento, de disminuci\'on o de desprecio (Aumentativo, Diminutivo o Despectivo). \\
\hline

Adjetivo & Masculino o Femenino, Singular o Plural, Determinativo o Calificativo; Demostrativo, Posesivo, Numeral, Relativo o Indefinido. Grado Positivo, Grado Comparativo o Grado Superlativo. Cuando portan Sufijo pueden considerarse como Aumentativos, Diminutivos o Despectivos.\\
\hline
Verbo & Persona, Singular o Plural, Tiempo, Modo, Voz, Conjugaci\'on, Forma Personal o No Personal. \\
\hline
Adverbio & Lugar, Modo, Tiempo, Cantidad, Intensidad, etc. Si modifica al verbo, o a otro adverbio. \\
\hline
Pronombre & Masculino o Femenino, Singular o Plural, Determinado o Indeterminado.\\
\hline
Art\'iculo & Lugar, Modo, Tiempo, Cantidad, Intensidad, etc. Si modifica al verbo, o a otro adverbio. \\
\hline
Conjunci\'on & Coordinada o Subordinada; Copulativa, Adversativa, Distributiva o Disyuntiva; Causal, Final, Comparativa, Concesiva, Consecutiva o Condicional.\\
\hline

\end{tabular}
\label{table:tabla1}
\caption{Categor\'ias de palabras}
\end{table}
\end{center}