% \IUref{IUAdmPS}{Administrar Planta de Selección}
% \IUref{IUModPS}{Modificar Planta de Selección}
% \IUref{IUEliPS}{Eliminar Planta de Selección}

% 


% Copie este bloque por cada caso de uso:
%-------------------------------------- COMIENZA descripción del caso de uso.

%\begin{UseCase}[archivo de imágen]{UCX}{Nombre del Caso de uso}{
	

	\begin{UseCase}{CU1}{Registrar Contacto}{El caso de uso permite al Contacto registrarse en el Sistema de mensajer\'ia para iniciar sesi\'on posteriormente y hacer uso de \'este.
	}
		\UCitem{Versi\'on}{0.1}
		\UCitem{Actor}{Contacto}
		\UCitem{Prop\'osito}{Registrar a un contacto en la base de datos del sistema de mensajer\'ia}
		\UCitem{Resumen}{
		El caso de uso permite al usuario crear un perfil a partir del llenado de campos validados dentro de un formulario}
		\UCitem{Entradas}{Nombre de Usuario:String  (obligatorio),
				 Contrase\~a:String (obligatorio),
				 Comprobar contrase\~na:String (obligatorio),
				 Correo el\'ectr\'onico:String (obligatorio), 
			         Nombre(s):String (obligatorio), 		
				 Apellidos:String (obligatorio),
				 Fecha de nacimiento: date(obligatorio),
				 Sexo: char(obligatorio}
		\UCitem{Salidas}{Mensaje de alerta que notifique al usuario que su perfil contacto ha sido creado}
		\UCitem{Precondiciones}{El usuario entra a la p\'agina de registro mediante la url de registro o haber sido redericcionado desde la pantalla de login}
		\UCitem{Postcondiciones}{Un nuevo Contacto debe haber sido agregado a la base de datos. . El Contacto registrado podr\'a iniciar sesi\'on en el sistema de mensajer\'ia, as\'i y podra acceder a su perfil de usuario}		
		\UCitem{Autor}{Anahi Ruiz Diaz}
	\end{UseCase}

	\begin{UCtrayectoria}{Principal}
	
		\UCpaso[\UCactor] El usuario entra a la p\'agina de registro del sistema

		\UCpaso  El sistema carga en la p\'agina un formulario con los siguientes campos de entrada: nombre de usuario, contrase\~a, comprobar contrase\~na, correo el\'ectr\'onico,nombre(s),apellidos,fecha de nacimiento,sexo, as\'i como un boton enviar.
		\UCpaso[\UCactor] El usuario ingresa los datos solicitados en el formulario.
		\UCpaso[\UCactor] El usuario da clic en el bot\'on \IUbutton{Enviar}.
                \UCpaso El sistema valida que todos los campos sean val\'idos y que no haya campos vac\'ios.\Trayref{A}
		\UCpaso El sistema verifica que el Nombre de usuario no se encuentre ya registrado anteriormente en su base de datos.\Trayref{C}
		\UCpaso  El sistema guarda los datos del nuevo Contacto a la base de datos. \Trayref{C}
		\UCpaso  El sistema muestra una pantalla de notificaci\'on al Contacto de registro exitoso.
		\UCpaso[] Fin del flujo.
				
	\end{UCtrayectoria}
		
		\begin{UCtrayectoriaA}{A}{Datos obligatorios}
			\UCpaso El sistema muestra una alerta que notifica al usuario que existen datos obligatorios que se encuentran vac\'ios.
			\UCpaso[\UCactor]  El usuario ingresa los datos faltantes solicitados por el sistema.
			\UCpaso[] El usuario da clic en el bot\'on \IUbutton{Aceptar} para concluir el registro.
			\UCpaso El sistema valida nuevamente que los datos solicitados obligatoriamente no se encuentren en blanco.
			\UCpaso[] Fin del flujo alterno.
		\end{UCtrayectoriaA}

		\begin{UCtrayectoriaA}{B}{Datos obligatorios}
			\UCpaso El sistema muestra una alerta que notifica al usuario que el nombre de usuario proporcionado ya existen dentro de la base de datos.
			\UCpaso[\UCactor]  El usuario ingresa otro nombre de usuario
			\UCpaso[] El usuario da clic en el bot\'on \IUbutton{Aceptar} para concluir el registro.
			\UCpaso El sistema valida nuevamente valida el nombre de usuario del contacto.
			\UCpaso[] Fin del flujo alterno.
		\end{UCtrayectoriaA}
		
		\begin{UCtrayectoriaA}{C}{Error}
			\UCpaso El sistema muestra un mensaje de alerta notificando al usuario que ha ocurrido un error y no ha sido posible registrarlo en el sistema.
			\UCpaso[\UCactor] El usuario da clic nuevamente en el bot\'on \IUbutton{Aceptar} para concluir el registro.
			\UCpaso El sistema ingresa nuevamente los datos del nuevo Contacto a la base de datos.
			\UCpaso[] Fin del flujo alterno.
			
		\end{UCtrayectoriaA}
		
		
%-------------------------------------- TERMINA descripción del caso de uso.
